\section{Muon $g-2$/EDM experiment at J-PARC}

\subsection{Principle of Measurements}\label{sec:Principle} 

The experiment aims to measure the muon's anomalous magnetic moment ($g-2$) 
and its electric dipole moment (EDM) at the J-PARC muon facility at MLF, MUSE.
The sensitivity goal is 0.46 parts per million (ppm) 
in the begining, aiming for 0.1~ppm as a ultimate goal.
The experiment would store $\mu^{+}$ in an MRI-type 
solenoid magnet and measure polarized $\mu^{+}$ decays to positrons.  
Higher energy $e^{+}$ are emitted with the muon spin direction 
aligned with its motion in the storage ring. Muon spin precession as 
the muons orbit in the storage ring result in the number of high energy 
$e^{+}$ oscillating as the muon spin precesses from forward to backward 
from the muon direction.  This precession, due to an advance of the 
muon spin precession compared to the muon momentum in a magnetic field, 
is due to the difference $g-2$.  Detectors on the inside 
of the muon storage orbit will track the decay
$e^{+}$, and the experiment collects the number of higher energy $e^{+}$ 
versus time in store, as the muon spin precesses in the magnetic field.
The $g-2$ experiment requires
precision measurements of the higher energy $e^{+}$ with time, and of the
magnetic field.

\begin{figure}[!hhh]
  \begin{center}
%      \includegraphics[width=15cm]{Fig/experiment_overview.pdf}
  \end{center}
  \caption{Overview of the muon $g-2$/EDM 
    experiment at J-PARC MLF. }
  \label{fig:g-2}
\end{figure}

The EDM measurement will be done with the same $\mu^{+}$ decays as the 
$g-2$ measurement. As described in details in the later sections, both 
dipole measurements can be done at the same time.
We propose to measure
the spin precession as a vector, specifically one component parallel
to the magnetic field, and the other component orthogonal to both
momentum vector and the magnetic field. The latter corresponds to 
the precession due to the muon EDM. 

The J-PARC muon $g-2$/EDM  measurement will use a very different 
approach compared to previous muon $g-2$ experiments.  The J-PARC
experiment will use an order of magnitude lower energy and an order of 
magnitude smaller diameter storage ring.  Previous experiment used 3.1~GeV/c
muons and a 14~m diameter storage ring, and
used calorimetry to measure the decay positrons.  The energy of the
previous experiments was set by a cancellation in the contribution to the 
spin precession from strong electric field focusing at this energy.  For
J-PARC, it is proposed to develop a source of ultra-slow muons followed by
acceleration, which require
only weak focusing to maintain the beam size in a storage ring.  This releases
the requirement on the muon energy, and a momentum 
of 300~MeV/$c$ was chosen with a 3~T
MRI-type solenoid used to store the muons.  Tracking detector will measure the 
decay $e^{+}$ momentum with time.

Previous experiments reached a precision on muon $g-2$ of 
0.54~ppm, with the uncertainty dominated by statistics.  The 
result, with the most recent and sensitive from E821 at Brookhaven National
Laboratory (BNL), is not described by the Standard Model of particle 
physics, with a $\sim$3 sigma deviation, suggesting the need for physics 
beyond the Standard Model.  
Muon $g-2$ is a fundamental observable of elementary particle and it is 
valuable to improve upon its
measurement and also to perform the experiment using different approaches.  
Any comprehensive theory of particle physics must describe muon $g-2$.  
We have proposed the J-PARC experiment to make a completely independent
measurement of muon $g-2$, with completely different experimental 
issues and systematics.  We also aim to improve the $g-2$ uncertainty 
from the BNL experiment by a factor 5 in the Phase 2. We note that a $g-2$ experiment 
is going forward at Fermilab using, in general, the same technique as at 
BNL, with 3.1~GeV muons and plans to transport the 14~m storage ring from 
BNL to Fermilab~\cite{FNAL_E989_TDR}. They aim to reach a sensitivity of 0.14~ppm on $g-2$.

We note here that the lower energy beam for the J-PARC experiment will
allow us to frequently reverse the muon spins.  This will be a powerful
tool to cancel difficult systematics such as pile-up which can change
detection efficiency versus time.  This spin reversal and has not been used 
before in previous $g-2$ experiments.  We also note that the spin reversal is beneficial 
in reducing the systematic uncertainties in the EDM measurement. 

The J-PARC experiment uses techniques that are completely new to the $g-2$
measurement which, while providing very different systematic uncertainty
issues compared with previous and new at Fermilab measurements, this
requires experimental tests of many new techniques. 
%We describe in this overview the approach
%, and indicate the areas that must be assessed further.
These tests are included in a list of detailed milestones that follow this
overview.  Details of each experimental step then follow the milestones.
