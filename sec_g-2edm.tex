\section{Muon $g-2$/EDM experiment at J-PARC}

\subsection{Principle of Measurements}\label{sec:Principle} 

The current experimental result for $a_{\mu}$ was from the E821 experiment 
at Brookhaven National Laboratory~\cite{Bennett:2006fi}, 
which used the ``magic gamma" approach with 100\% polarized 3~GeV muons 
injected by an inflector magnet with 2-5\% efficiency into a 14 meter diameter 
storage ring built with 360~degree superconducting coils, 
12 iron back-leg sectors and 36 iron pole sectors. 
With iron shims, a 1 part per million (ppm) field uniformity was achieved after
averaged over the muon orbit, with local non-uniformity of up to 100~ppm. 
Electrostatic focusing was used in the ring, and decay positrons (and electrons) 
were observed with calorimetry.  A new measurement of $a_{\mu}$ is underway 
at Fermilab~\cite{Grange:2015fou}, using the BNL-E821 14~meter diameter storage ring,
with a new muon accumulator ring and significant magnetic shimming improvements, 
with expected improvements in statistical and systematic uncertainties.

The experiment introduced here
is intended to measure $a_{\mu}$ and $d_{\mu}$ with a very different technique,
using 300~MeV/$c$ reaccelerated thermal muon beam with 50\% polarization,
vertically injected into a MRI-type 66~cm diameter solenoid storage ring 
with 1~ppm local uniformity for the storage magnetic field.
The vertical injection, invented for this experiment, 
will improve injection efficiency by more than an order of magnitude.
Very weak magnetic focusing will be used in the ring. 
Silicon strip detectors in the field will measure 
the momentum vector of the decay positrons. 

Table~\ref{T:Comparison} compares the proposed experiment with the previous
experiment BNL-E821, and the current experiment Fermilab-E989.
The initial goal of this experiment is to reach the statistical uncertainty for $a_{\mu}$ 
of BNL-E821, with much smaller systematic uncertainties from sources different from the current method.
The muon EDM goal is a statistical sensitivity of $1.4\times 10^{-21}~e\cdot\mbox{cm}$
with a systematic uncertainties of $0.36\times 10^{-21}~e\cdot\mbox{cm}$,
which is a factor 60 improvement over the present limit~\cite{Bennett:2008dy}.

The experiment measures $a_{\mu}$ and $d_{\mu}$. They are defined as 
\begin{equation}
a_{\mu} = \frac{g-2}{2} ~~~ {\rm with} ~~~~ \vec{\mu} = g \left( \frac{e}{2m}
                  \right) \vec{s}, ~~~~
\vec{d} = \eta \left( \frac{e}{2mc}
                  \right) \vec{s},
\end{equation}
where $e, m$ and $\vec{s}$ are the electric charge, mass, and spin vector of muon, respectively.
Here, $g$ is the Land\'e's $g$ factor and $\eta$ is a corresponding factor for EDM.
The experiment stores spin polarized $\mu^{+}$ in a magnet and 
the muons orbit under the uniform magnetic field. 
The spin of the muon precesses in magnetic field. 
With the non-zero and positive value for $g-2$,
%, the spin precession advances the rotation of the momentum direction. As a result, 
the muon spin direction makes a slow rotation 
with respect to the momentum direction. 

The spin precession vector with respect to its momentum 
in static magnetic field $\vec{B}$ and electric field $\vec{E}$
is given as~\cite{Fukuyama:2016xpk, Fukuyama:2016zsi, Silenko:2017iyv},
\begin{eqnarray} 
\vec{\omega} & = & \vec{\omega}_{a} + \vec{\omega}_{\eta} \\ 
             & = & 
  - \frac{e}{m}\left[a_{\mu} \vec{B} - 
   \left( a_{\mu}- \frac{1}{\gamma^2-1} \right)
   \frac{\vec{\beta} \times \vec{E}}{c}
  +\frac{\eta}{2} 
   \left(\vec{\beta} \times \vec{B}  + \frac{\vec{E}}{c}
   \right)
   \right].
\label{eq:omega_full}
\end{eqnarray}
Here $\vec{\omega}_{a}$ and $\vec{\omega}_{\eta}$ are precession vectors
due to $g-2$ and EDM. $\vec{\beta}$ and $\gamma$ are velocity and Lorentz factor of muon,
respectively.

In the previous $g-2$ measurements, the energy of the muon
was chosen to cancel the term of $\vec{\beta} \times \vec{E}$,
which allowed for electrostatic focusing in the storage ring 
without affecting the muon spin precession to first order. 
A focusing field index of $n =$0.12--0.14 was used, 
which was necessary to contain the muons captured from pion decay. 
In this experiment, we greatly reduce the focusing requirement 
in the storage ring by using the reaccelerated thermal muon beam with a factor of 1,000 smaller beam emittance.
A very weak magnetic focusing with a field index of $n \sim 10^{-4}$
is enough to store the muon beam with no electric field.
Under this condition, Eq.~(\ref{eq:omega_full}) reduces to
\begin{equation}
\vec{\omega}  =  
  - \frac{e}{m}\left[a_{\mu} \vec{B} 
  +\frac{\eta}{2} 
   \left(\vec{\beta} \times \vec{B} \right) 
   \right].
\label{eq:omega_tot}
\end{equation}
There is no contribution from the $\vec{\beta} \times \vec{E}$ term at any beam energy.
Since the precession vectors $\vec{\omega}_{a}$ and $\vec{\omega}_{\eta}$ 
are orthogonal each other, the $g-2$ and EDM precessions 
can be measured simultaneously with
an appropriate detector design.

The key requirement for this new approach to the measurement is a muon beam with low emittance.
This can be realized by a source of 
thermal energy positive muons followed by
reacceleration, without changing the transerse momentum spread. 
We note here that the stopping muons and their reacceleration steps will also
allow us to frequently reverse the muon spins by using static electro-magnetic fields.
This feature will be a powerful tool to study
rate-dependent systematics such as track reconstruction efficiency, and the effect on pile-up hits.

In the extraction of $a_{\mu}$ and $\eta$, 
the precession frequency $\vec{\omega}$ and the magnetic field $B$ must
be measured. The $\vec{\omega}$ is measured by detecting positrons from muon decay during the storage.
Since higher energy $e^+$ are emitted with the muon spin direction 
aligned with its motion~\cite{Konopinski:1959qr}, the number of high energy $e^+$ will show 
an oscillation in time as the muon spin precesses from forward to backward 
with respect to the muon momentum direction.
Detectors located radially inside the muon storage orbit will track the decay
$e^{+}$. The experiment records the number of higher energy $e^{+}$ 
versus time in storage, as the muon spin precesses in the magnetic field.

The average magnetic field seen by the muons in the storage ring 
is measured by the Larmor precession frequency of a free proton ($\omega_p$).
This is obtained from a convolution of the magnetic field map
and the muon beam distribution measured by the experiment.

Assuming the EDM term is negligibly small compared with the $g-2$ term in Eq.~(\ref{eq:omega_tot}), 
$a_\mu$ is obtained from 
$\omega_a = \frac{e}{m} a_\mu B$.
By using $\omega_p$, one can rewrite this equation to
\begin{eqnarray}
a_\mu = \frac{R}{\lambda - R},
\label{eq:amu}
\end{eqnarray}
where $R = \omega_a /\omega_p$ and $\lambda = \mu_\mu/\mu_p$ is the muon-to-proton
magnetic moment ratio externally provided.
The precision of the direct measurement of $\lambda$ by muonium spectroscopy in magnetic field
is 120~ppb~\cite{Liu:1999iz}. 
A new improved measurement of $\lambda$ is being prepared at J-PARC MLF in the same beamline~\cite{Shimomura:2015aza}.

\begin{figure}[t]
  \begin{center}
      \includegraphics[width=15cm,bb=0 0 1440 936]{Fig/g-2edm_overview_en.pdf}
  \end{center}
  \caption{Schematic of the muon $g-2$/EDM experiment at J-PARC MLF~\cite{TDRsummarypaper}. }
  \label{fig:setup}
\end{figure}

The experiment will be installed at the muon facility~\cite{Higemoto:2017} in
the Materials and Life science Facility (MLF) of J-PARC.
A schematic of the experimental setup is shown in Fig.~\ref{fig:setup}. 
Experimental components and sensitivity estimations are described in the following sections.

%====================================================
\subsection{Experimental facility and surface muon beam}
A 3~GeV primary proton beam with 1~MW beam power from the Rapid Cycle Synchrotron hits 
a 2~cm thick graphite target to provide pulsed muon beams.
The proton beam has a double-pulse structure, and each pulse is 100~ns in FWHM 
with a 600~ns separation and 25~Hz repetition rate.
The experiment uses a surface muon beam.
Surface muons are 100\% polarized positive muons from pions 
stopped at and near the target surface with the consequent momentum of 29.8 MeV/$c$ and below.
There are four beamlines extracting muon beams.
The experiment will use one of those, the H-line.

The H-line is a new beamline designed 
to deliver a high intensity muon beam~\cite{Kawamura:2014hja}.
This is realized by 
adopting a large aperture solenoid magnet to capture muons from the muon production target,
wide gap bending magnets for momentum selection,
and a pair of different directional solenoid magnets for efficient beam transport.
The surface muon beam is focused onto a target to produce muonium atoms.
The final focus condition is optimized to maximize the number of muons stopping in the muonium production 
target and to minimize the leakage magnetic field at the focal point. To fullfill these 
requirements, 
%the final focusing magnets are composed of a solenoid magnet
the final focusing includes a solenoid magnet
followed by a triplet of quadrupole magnets.

The intensity of the surface muon beam at H-line is estimated 
to $10^8$ per second at the designed proton beam power of 1 MW.
The surface muon at the end of the beamline has a momentum centered at 28.2 MeV/c with 
10\% momentum bite. According to a beam transport simulation
the beam size at the focal point will be $\sigma_x=31$~mm and $\sigma_y=14$~mm.

%===============================================================
\subsection{Production of thermal muons from surface muons} 
%
The surface muon beam is converted at its final focus into a
source of room-temperature muons. The first step is to
slow and thermalize the $\mu^{+}$ in a carefully selected
material, silica aerogel \cite{Tabata:2011aa}.  In this
material, most of the muons form muonium atoms ($\mu^{+}e^{-}$, or Mu)
that diffuses as a neutral atom into a vacuum region where Mu is
ionized by laser excitation (Fig.~\ref{fig:usm}). While the thermalization, conversion
to Mu, diffusion, and ionization steps result in the loss of a significant
fraction of the original surface muon beam, the
characteristics of thermal muons after muonium ionization can be
exploited as a source for acceleration and injection into a
storage ring. A comparison of the kinematic characteristics of
surface muons, a thermal source, and accelerated muon
is summarized in Fig.~\ref{fig:usm}.

\begin{figure}[t]
\centering
\includegraphics[width=1.0\columnwidth, bb=0 0 680 312]{Fig/muonbeam.pdf}
\caption{Scheme of the reaccelerated thermal muon beam~\cite{TDRsummarypaper}.
The surface muon beam is stopped in a silica aerogel plate at 
the downstream edge. Some of the formed muoniums diffuse to surface 
and evaporate to vacuum with thermal energy. 
Intense laser beams strip the electron from muonium 
and the muon is accelerated by a static electric field followed by 
RF acceleration. 
}
\label{fig:usm} 
\end{figure}

Silica aerogel is chosen as the muonium production target for
high Mu formation probability ($>0.5$) and low
relaxation of the polarization.
The 50\% polarization is the maximum after statistical spin
distribution among hyperfine states in the Mu atom.
In addition, the silica aerogel provides the large mobility of
Mu atoms within the aerogel structure such that they can be
emitted with a near-thermal room temperature energy distribution
from the surface of the slab into the adjacent vacuum region.

The emission of Mu from aerogel, as well as the other important
characteristics described above, has been discovered and verified by experiments
on a surface muon beam line at TRIUMF~\cite{Bakule:2013poa,Beer:2014ooa} and J-PARC.
The results showed that the
emission probability was enhanced by an order of magnitude if the
downstream aerogel surface was covered with a close-packed array of holes
produced by laser ablation to a depth of order a few mm. 
The data are consistent with the assumption of Mu diffusion 
within the aerogel slab to the surface of ablation holes 
followed by emission through the holes 
with speeds corresponding to thermal velocity near
room temperature.

The evolution of muonium into the laser irradiation region
located at 1~mm from the surface of the aerogel slab is simulated.
The laser region is defined as a volume of $40\times 40\times 5$~mm$^3$ in the transverse directions,
and the longidudinal direction, respectively.
This simulation indicates that the optimum time for the
short ionization pulse is near 1.0 $\mu$s after the
average time of arrival of the two surface muon pulses (0.6~$\mu$s apart).
The efficiency for thermal muonium production in the laser region 
is estimated as $3.8\times 10^{-3}$ per surface muon.

A high-power ionizing laser beam system is synchronized to the
periodic 25~Hz thermal Mu production at its maximum density in vacuum.
One laser at 122~nm (Ly-$\alpha$) with 80~GHz linewidth and 100~$\mu$J per pulse excites the Mu from the $1s$ atomic state
to $2p$, then a second at 355~nm with 300~mJ strips the electron. The
laser pulses are 1~ns duration. The estimated ionization efficiency of this process is 73\%.
The generation of coherent Ly-$\alpha$ light is realized by using a nonlinear conversion process 
of the two-photon resonance four-wave difference frequency mixing in Kr gas.
The feed lights are generated by a solid state laser system consisting of 
a fiber coupled distributed feedback laser diode followed by
three stages of amplification and nonlinear conversions.
Such an intense Lyman-$\alpha$ laser \cite{laser:2016} is being developed 
in collaboration with the group developing an ultra slow muon microscope, which 
is being used for the ionization of muonium at J-PARC U-line \cite{usm:2018}. 


%=========================================================
\subsection{Acceleration}

%%%%%%%%%%
%Overview
%%%%%%%%%%

The room-temperature muons created in the laser ionization of thermal muonium will be 
accelerated to a momentum of 300~MeV/$c$ (212~MeV in kinetic energy).
The muons must be accelerated in a sufficiently short time
compared with the muon life time of 2.2~$\mu$s
to suppress muon decay loss during the acceleration.
Another essential requirement for the acceleration is
the suppression of transverse emittance growth.
To satisfy these, a linac dedicated to this purpose will be used in the experiment.
Figure~\ref{fig:mulinac_config} shows the schematic configuration
of the muon linac.
In accelerating the muons, the $\beta$ increases rapidly with the
kinetic energy.
It is important to adopt adequate 
accelerating structures to obtain high acceleration efficiency,
similar to proton linacs. 
The acceleration steps are 1) electrostatic acceleration with a Soa lens,
2) radio frequency quadrupole (RFQ),
3) inter-digital H-type drift tube linac (IH-DTL),
4) disk-and-washer structure (DAW),
and 5) disk-loaded traveling wave structure (DLS).

\begin{figure}[t]
\centering
\includegraphics*[width=1.0\textwidth,bb=0 0 680 170]{Fig/muonlinac.pdf}
\caption{Schematic configuration of the muon linac~\cite{TDRsummarypaper}.}
\label{fig:mulinac_config}
\end{figure}

%%%%%%%%%%%
% Init Acc
%%%%%%%%%%%

 As the first acceleration step, thermal muons are accelerated from the ionization region
by a pair of meshed metal plates and an electro-static lens, a Soa lens~\cite{Soa}.
The ellipses in the $x$-$x^{\prime}$ and $y$-$y^{\prime}$ distributions represent
the matched ellipses of 1.0$\pi$~mm~mrad.
The lower right figure represents the time structure at the entrance
of the RFQ.
Even though the pulse width of the dissociation laser is 1~ns,
the time width at the RFQ entrance is 10~ns
owing to the spatial distribution at ionization.
Therefore, the beam from the source divides into three bunches during 
the acceeleration in the RFQ at the frequency of 324~MHz.
A spare RFQ of the J-PARC linac~\cite{kondo:hptest_rfqII:prstab2013}
will be used as a front-end accelerating
structure accelerating the muons to 0.34~keV~\cite{kondo:simulation_mu_rfq:ipac2015}.

%%%%%
% IH
%%%%% 
\newcommand{\Ez}{E_{z}}

The energy of the muon beam is boosted to 4.5~MeV with an IH-DTL.
Different from the Alvarez DTL, the IH-DTL uses the TE11 eigenmode,
and $\pi$-mode acceleration~\cite{bib:IH_blewette}. 
With this mode, the acceleration length is halved compared with
the 2$\pi$-mode acceleration. 
In addition, alternative phase focusing (APF)~\cite{bib:IH_APF1} is adopted.
Since the use of APF eliminates the need for installing quadrupoles
in the drift tubes, a higher shunt impedance per length can be achieved. 
The beam dynamics with such an IH-DTL was studied~\cite{Otani:2016swo}.
Sixteen cells are required to accelerate up to 4.5~MeV,
and the total length of the cells is 1.29~m. 
The quality factor $Q_{0}$ is calculated to be $1.03\times 10^{4}$, 
and the power dissipation is 320~kW. 
The effective shunt impedance per unit length is calculated
to be 58~$\mathrm{M\Omega/m}$,
which is competitive with those of other IH structures, taking 
our IH application to a relatively higher velocity region into account. 

%%%%%%
% DAW
%%%%%%

Following the IH-DTL, DAW structures
with a frequency of 1,296~MHz are used accelerate to 40~MeV.
The DAW is one of the coupled-cavity linacs which has large coupling 
between the cells and a high shunt impedance, especially in the middle
$\beta$ section~\cite{andreev:he_proton_linac_structures:linac1972}. 
The cell design was optimized for the velocities of 
$\beta=0.3, 0.4, 0.5,$ and $0.6$ 
by using the SIMPLEX algorithm~\cite{otani:develop_mulinac:ipac2016}.
PARMILA~\cite{parmila} was used to design
the beam dynamics of the DAW section.
The acceleration gradient is determined to be 5.6~MV/m so as to keep
the maximum electric field less than 0.9 times the Kilpatrick
limit~\cite{bib:kilpatrick}. 
The field strength of the quadrupole doublet between the modules and
the number of the cells in each module are determined
with a condition that the phase advance in one focusing period
is less than 90~degrees. 
The number of the cells in a module is set to ten, and  
the phase advance is approximately 83~degrees in the first module,
where the RF defocusing is strongest. 
The total length is 16.3~m with 15~modules.
The estimated power dissipation is 4.5~MW. 

%%%%%%
% DLS
%%%%%%

 Finally, the muons are accelerated from 40~MeV to 212~MeV by using a DLS,
which are widely used for electron linacs.
 The advantage of the DLS is its high acceleration gradient;
approximately 20~MV/m.
 An RF frequency of 1,296~MHz is adequate for the wider phase space.
The particular design feature of the DLS for
the muon acceleration, which is different from 
the general accelerating structure for the electron accelerator, 
is the variation of the disk spacing corresponding to 
the muon velocity~\cite{kondo:beamdynamics_muon_highbeta:ipac2017}.
 The DLS section consists of four accelerating structures and 
the total length is approximately 10~m.
The estimated momentum spread is 
0.04~\%(RMS).

With this design of the muon linac, these simulations show that the transmission efficiency is kept high,
and there is no significant growth of the beam emittance during the acceleration.

%===============================================================
\subsection{Beam injection and muon storage magnet}

The muon beam must be injected into the muon storage magnet and the injection system
must have minimum interference to the storage field. For reasons described later,
a new method to inject the muon beam from the top of the magnet is adopted.
After the linac, the muon beam follows a beam transport line to inject the muon 
beam at an incident vertical angle of 25~degrees. The beam transport line consists of 
two dipole magnets for bending the beam vertically, three normal quadrupole magnets
to match the vertical momentum dispersion and eight rotated quadrupole magnets
to control the phasespace to match the acceptance of injection into the magnet.

\begin{figure}[t]
%\vspace*{-0.3cm}
%\vspace*{7.0cm}
 \centerline{\includegraphics[width=12cm]{Fig/MagnetMechOverviewBW.eps}}
 \caption{Overview of the muon storage magnet~\cite{TDRsummarypaper}.}
\label{fig:magdesign}
%\vspace*{-0.3cm}
\end{figure}

A $3$~T MRI-type superconducting solenoid magnet will be used to complete the injection and store the muon beam.
Figure~\ref{fig:magdesign} shows an overview of the muon storage magnet~\cite{sasaki:2016}.
The muons are stored in a 3~T magnetic field with a cyclotron radius of $333$~mm.
The size of the storage magnet is about a factor of 20 smaller than the magnet for BNL/Fermilab experiments.
We take advantage of the advance in MRI magnet technology to fabricate such a small storage magnet with a highly uniform magnetic field in the muon storage region.
The magnet system has four functions: (1) provide a high uniform storage field; 
(2) provide the injection field; (3) provide the kicked field to store the muons; 
and (4) provide weak focusing for storage.

The main feature is a highly homogeneous magnetic field of $3$~T (main field) 
in the central region of the magnet, the storage region, where the muon beam is stored until its decay. 
The homogeneity of magnetic field in the storage region is directly related to the sensitivity of the $a_{\mu}$ measurement.
The integrated magnetic main field uniformity along the beam orbit in the storage region has to
be carefully controlled with a precision of $100$~ppb peak-to-peak.
The relative field distribution in the r-z plane around the storage region, averaged over the muon orbit has a variation less than $\pm 50$~ppb. 
This field uniformity is a feature of this experiment; the field variation along the muon orbit for the BNL~(E$821$) magnet was as large as $\pm$~100 ppm.

The second function is to transport the muon beam from the outside of the storage magnet to the storage region. 
This transportation region is named the injection region.
Due to the limited space of the storage magnet, the muon beam is not injected by the previous method used for CERN, BNL and Fermilab $g-2$ experiments of horizontal injection using an inflector magnet with poor ($2\sim5\%$) efficiency. 
Instead, a new $3$-D spiral injection scheme~\cite{Iinuma:2016zfu} is developed for this purpose.

A solenoid magnetic field shape is suitable for this new injection scheme. 
In the injection region, the radial component, $B_{r}$, of the magnetic field has to be carefully controlled from the top end of the magnet to the storage region for smooth injection. 
A three-dimensional view of beam trajectories from the injection region to the storage region is also shown in the right side. 
The muon beam enters from a downward with injection angle (pitch angle) of $440$~mrad. 

Open circles along the beam indicate points, which correspond to the radial field values on the left side.  
The beam momentum is deflected by the radial component of the fringe field $B_{r}$ 
as it reaches the mid plane of the solenoid magnet. Within the first three turns, the pitch angle becomes $40$~mrad.  
We design the fringe field to control beam vertical motion.
And at the same time, this fringe field requires  
appropriate vertical-horizontal coupling (so-called $X$-$Y$~coupling in the beam coordinate) to control vertical divergence, 
because of an axial symmetric shape in the fringe field. 
The $X$-$Y$~coupling of the beam phase space, controlled by the magnet staying just upstream of the solenoid, will be carefully tuned to 
minimize the vertical beam size in the storage region. 

The third function of the magnet system is to provide a vertical kick, which will guide the beam inside the storage region.
Two pairs of one-turn coils, the kicker coils at $\pm0.4$~m in height, 
generate a pulsed radial field $B_{kick}$ to apply a vertical kick to the muon beam motion. 
The vertical beam motion from the start of the kick to the end, as well as the beam motion in the storage region.

In order to keep the beam inside of the storage region within a stable orbit,
a weak focusing magnetic field~\cite{Iinuma:2016zfu,Abe:2018tmp} will be used.
The equations of the weak focusing magnetic field are 
\begin{eqnarray}
 B_r &=& -n\frac{B_{0z}}{R}z, \\
 B_z &=& B_{0z}-n\frac{B_{0z}}{R}(r-R)+ n\frac{B_{0z}}{2R^2}z^2,  
\label{eq:weak}
\end{eqnarray}
where, $B_{0z}$ (3~T) is the field strength in the $z$ direction at the center of the storage region,
$R$ (333~mm) is the average radius of the stored beam, $n$ is the field index.
The weak focusing field is the fourth function of the magnet system.

%%%%%%
%Finally, we introduce mechanical design of the magnet system briefly. 
The solenoid will be composed of 5 main coils wound with NbTi cable
and the inner radius will be 0.8~m in the present design.
%Shim coils for compensating field errors caused by the coil misalignment and so on, are installed
%on the outside of the main coils.
An iron yoke is used to suppress a magnetic flux leakage.
The magnet has pole tips at the both ends of the solenoid coil to form the magnetic flux, with an entrance hole for injection.

The main coils will be operated with a persistent current mode with a superconducting switch.
%in order to eliminate a fluctuation of the magnetic field due to an instability of current source.
The time constant of current decay is generally expected to be less than 10 ppb/hour.
This field decay will be constantly monitored by the NMR probes during the measurement.

The weak focusing magnetic field is generated by dedicated coils, the weak focus coils, consisting
 of 8 ring coils wound with NbTi cable which are aligned in the axial direction of the magnet.
All ring coils are connected in series electrically and driven by a single power supply.

The magnetic field is shimmed by passive and active shimming systems.
The former uses iron pieces,
which are attached on support cylinders installed inside the magnet bore through holes in the iron poles in air.
The magnetic field distribution is adjusted by changing the alignment pattern of iron pieces.
Active shimming is done using a superconducting shim coils wound with NbTi cable.
They are mainly used to compensate the error field changing with time in the storage region,
and the residual error (expected to be small) after the magnetic field shimming by iron pieces.
The shim coils consist of several saddle coils which have a four fold symmetry.
Each coil is connected to independent power supply to control each current.

The main coils, the weak focus coils, and the shim coils are immersed in liquid helium to ensure good temperature stability.
%The temperature could be controlled with a precise pressure control system of the helium vessel.
The helium is re-condensed by cryocoolers for long-term, stable and cost-effective operation.
Four cryocoolers and a heat exchanger for the helium re-condensation will be installed in a cold box,
 placed apart from the magnet cryostat.
A connection pipe between the cold box and the magnet cryostat has a bellows connection,
which is a soft connection in terms of mechanical structure, so that
the vibration of cryocoolers will not be directly transferred to the magnet.

The magnetic field is measured by a system with the NMR (nuclear magnetic resonance) method.
To obtain the average magnetic field $\omega_{p}$ experienced by the muon beam,
the magnetic field is measured along the muon storage orbit (integrated) with a precision below 100 ppb level
in the main field of 3~T where local field homogeneity is 1,000 ppb level.
The NMR is the only solution with this precision.

A continuous wave type (CW) NMR will be used in this experiment. 
The resonant absorption signals of protons in water samples is observed
by using a fixed frequency source and a small sweeping magnetic field.
The NMR probe will have a size of about 5 -- 10~mm in diameter.
The magnetic field in the storage region will be mapped by scanning the probe with moving stages.
The stages are driven by ultrasonic motors, which can work in the strong magnetic field.
The ultrasonic motors have encoders so that the position of NMR probe is controlled
with a precision of below 0.1~mm.

%===============================================================
\subsection{Positron Detector}\label{sec:detector}

The positron detector is installed inside of the storage magnet
and measures positron tracks from decay of stored muon beam.
A muon with the momentum of 300~MeV/$c$ circulates with a radius
of 333~mm and decays to a positron, a neutrino and an anti-neutrino
with the life time of 6.6~$\mu$s. The cyclotron period is 7.4~ns.
Since the anomalous precession period is 2.2~$\mu$s, muons circulate
the ring about 300 times on average during one revolution of muon spin.
The goals of the detector are to measure $\omega_{a}$
and the up-down asymmetry of positron direction due to EDM.

%Since the muon decay breaks the parity, high-energy positrons in their decay
Since the muon decay breaks parity, high-energy decay positrons tend to be emitted
in the direction of muon spin~\cite{Michel}. By measuring high energy positrons
selectively, positrons emitted forward can be selected and
the time variation of muon spin with respect to the muon momentum direction
can be measured. The sensitivity becomes maximum when positrons
with momentum above 200~MeV/$c$ are counted.

Positrons emitted within the 3~T magnetic field move in a spiral orbit.
This trajectory is detected by radially arranged silicon strip sensors.
Geometrical coverage of the detector is ***--290~mm in radial direction and within $\pm$400~mm
in height. Layout of the detector is shown in 
Fig.~\ref{fig:Detector_overview}. 

\begin{figure}[t]
  \begin{center}
    \includegraphics[width=0.4\textwidth, bb=0 0 393 511]{Fig/Detector_cutview.png}
    \includegraphics[width=0.4\textwidth, bb=0 0 337 369]{Fig/Detector_topview.png}
    \caption{Perspective view (left) and top view (right) of the positron detector~\cite{TDRsummarypaper}.}
    \label{fig:Detector_overview}
  \end{center}
\end{figure}

Muon beam spill rate is 25~Hz and the number of muons per spill is about $10^4$.
The measurement will be performed in the period of 33~$\mu$s, which is
 five times larger than the muon life time. The rate of positrons
changes by a factor of 160 from the beginning to the end of the measurement.
Thus, the detector is required to be stable against the change of positron rate;
otherwise, the measured $\omega_a$ would be biased.

The detector consists of 40 radial modules called vanes.
Each vane consists of 16 sensors.
Sensors are made by single-sided p-on-n technology~\cite{sensor}.
The active area of a sensor is 97.28 mm $\times$ 97.28 mm with
a thickness of 0.32~mm.
A sensor has two blocks of 512 strips and the number of strips
with 190~$\mu$m pitch. Therefore a vane has 16,384 strips, with 655k total strips for the detector.

The data from the silicon strip sensors are read out by front-end
boards on the detector with a 5~ns time stamp, followed by readout boards with 
VME interface,
then collected by the PC farm through a Gibabit Ethernet switch.
%The data size of one hit is 8 bytes when the leading and trailing edges are separately recorded.
The data acquisition system based on DAQ-Middleware~\cite{DAQ-Middleware} is used.
Estimated size of data from the whole detector is 360~MB/s (or 14.4~MB/spill).

One readout ASIC has 128 channels for analog block and digital block.
Dynamic range of input charge is required to be greater than
4 minimum ionizing particle (MIP) equivalent with linearity.
Equivalent noise charge is required to be less than 
1600~e$^{-}$ with the input capacitance of 30~pF, which
corresponds to the signal-to-noise ratio greater than 15
for a 1~MIP signal. Pulse width at 1~MIP charge is required to
be less than 50~ns and the time-walk between 0.5~MIP and 3~MIP
is required to be less than 5~ns to constrain a timing shift effect
due to pileup hits.

The system clock is provided by the GPS-synchronized Rb frequency
standard~\cite{freqtime}, and it is distributed with real time control signals
to the readout boards and the front-end board through the timing
control/monitor board. Long term stability of the system clock
frequency is confirmed better than $10^{-11}$.

The stringent requirement on the detector alignment comes
from the EDM measurement~\cite{EDM_requirement}. Alignment accuracies of vanes
with respect to the magnetic field direction
are required to be better than 10~$\mu$rad for skew, that is an
angle around an axis normal to the vane. 
In order to ensure the required accuracies, alignment changes for the
vanes are detected and monitored during operation using an absolute
distance interferometer system.

At the beginning of spill, about 30 positrons are produced from muon decay in 5~ns which is one time window of the data taking.
The maximum hit rate per silicon sensor strip is $7 \times 10^{-3}$ per time stamp.
To find positron tracks in such a condition,
a positron track candidate is searched from hits on the detector
using the property that high momentum positron tracks leave
nearly straight lines in the $\phi$-$z$ plane, 
where the $z$-axis is the center axis of the detector along the direction of magnetic field and $\phi$ is the angle around $z$-axis.
In the $\phi$-$z$ plane, (bottom right) straight lines used as seeds for track finding are shown.
Hough transformation~\cite{HoughTransform,UseHough} is used
to find straight lines in the plane and hits on a straight line are used as the seed. 
Track momentum is obtained
by track fitting with a Kalman filter~\cite{KalmanFilter}.
With this algorithm, track reconstruction efficiency
greater than 90\% is achieved in the positron energy range of
200~MeV$<E<$275~MeV even at the highest positron rate.

The muon decay position is
determined by the closest point of approach between the reconstructed positron
trajectory and the muon beam orbit.
The muon decay time at the decay position is measured by extrapolating the time of hits in reconstructed
positron tracks. 
One way to estimate decay time is to use the average time of reconstructed 
track hits.
An other approach way is to use the transition timing of hits with the 5~ns time stamp.
The latter method has better timing resolution than the former but
it is applicable only when the transition occurs within a track.
Two definitions of decay time can be cross-checked with each other.

%===============================================================
\subsection{Estimation of number of reconstructed positron}\label{sec:Intensity} 
Efficiencies of steps from the surface muon production to the detection of positrons are studied 
by a chain of simulations.
The simulations include the surface muon production,
thermal muon production, reacceleration, injection to the muon storage
magnet, muon beam dynamics in storage, and ended by the detection of the positron.
The simulation of surface muon production~\cite{otani:simulation_usm_production:ipac2018}
and thermal muon production are optimized by the experimental data
on surface muon yield at the existing beamline and measurements of muonium space-time distribution~\cite{Beer:2014ooa},
respectively.
Total efficiency is $1.4 \times 10^{-5}$ per initial muon at production.
At the proton beam power of 1~MW, the expected number of positrons is $5.7 \times 10^{11}$ for $2 \times 10^7$ seconds data taking.

\subsection{Extraction of $a_{\mu}$ and EDM}\label{sec:Sensitivities} 

The $\omega_a$ and $\eta$ are obtained from the muon decay time distribution.
The muon decay time is reconstructed from the positron track as described in Sec.~\ref{sec:detector}.
A simulated time spectrum for detected positrons in the energy range 
between 200~MeV and 275~MeV is shown in Fig.~\ref{fig:Wiggle}~(left).
The anomalous precession frequency $\omega_a$ is extracted by fitting to the data. 
Alternatively, one can make a ratio of 
data taken with different initial spin orientations.
This will be useful to study early-to-late changes in the detector performance.

\begin{figure}[t]
  \centering
    \begin{tabular}{c}

      \begin{minipage}{0.5\hsize}
        \centering
        \includegraphics[width=0.7\linewidth, angle=270, bb=20 255 428 822]{Fig/WigglePlot.pdf}
      \end{minipage}

      \begin{minipage}{0.5\hsize}
        \centering
        \includegraphics[width=0.7\linewidth, angle=270, bb=20 255 428 822]{Fig/EDMModuloPlot_20_40.pdf}
      \end{minipage}
    \end{tabular}

    \caption{Simulated time distribution of reconstructed positron (left) 
      and the up-down asymmery as a function of time modulo of the $g-2$ period (right).
      Solid curve is the fit to data~\cite{TDRsummarypaper}.}
    \label{fig:Wiggle}
\end{figure}

The $\omega_p$, average magnetic field seen by the muons in the storage ring, is 
measured by independent measurements of the magnetic field map in the storage ring provided from the proton
NMR data and the muon beam distribution deduced from tracing back the positron track to the muon beam.
A blind analysis will be done as was done by the previous BNL expriment, separating the results for 
magnetic field and sipin precession until all systematic uncertainties are finalized.

Once the $\omega_a$ and $\omega_p$ are extracted from the experimental
data, $a_\mu$ is obtained from Eq.~\ref{eq:amu}.
Table~\ref{tab:sensitivity} summarizes statistics and uncertainties for $2 \times 10^7$ seconds of data taking.
The estimated statistical uncertainties on $\omega_a$ and $\omega_p$ are 450~ppb and 100~ppb,
respectively. Thus, the statistical uncertainty of $a_{\mu}$ would be 460~ppb.

Systematic uncertainties on $\omega_a$ are estimated as follows.
A timing shift due to pile up of hits in the tracking detector is estimated as less than 36~ppb
in the detector simulation by taking into account time responses of readout electronics.
A correction for pitch angle is not necessary in the case of the muon storage 
in the perfect weak magnetic focusing field~\cite{Semertzidis:2016kte}. 
A difference in the actual field distribution 
from the perfect case leads to a systematic uncertainty of 13~ppb which is estimated from a precision spin-tracking simulation
of the muon beam storage.
Residual electric field modifies $\omega_a$ through the $\beta \times E$ term. 
With 1~mV/cm monitoring resolution for an E-field, the error on $\omega_a$ is 10~ppb. 
Other effects, such as delayed high-energy positrons and differential decay, are 
of the order 1~ppb.
 In the $\omega_p$ measurement, absolute calibration of the standard probe has an uncertainty of 25 ppb.
Positioning resolution of the field mapping probe at the calibration point and the muon storage
region leads to 20~ppb and 45~ppb uncertainties, respectively.
Other effects, such as field decay and eddy current from kicker are less than 10~ppb.
In summary, we estimate that the combined systematic uncertainties on $a_{\mu}$ is less than 70~ppb.

\begin{table}[t]
  \caption{Summary of statistics and uncertainties}
  \label{tab:sensitivity}
  \begin{center}
%  \begin{small}
  \begin{tabular}{|p{0.5\textwidth}|c|c}
  \hline
          & Estimation \\
    \hline
    \hline
%    Running time [s] & $2\times 10^{7}$ \\
%    Muon beam polarization & 0.5 \\ 
%    Average muon rate in the storage magnet [s$^{-1}$] & $2.6\times 10^{5}$ \\
    Total number of muons in the storage magnet & $5.2 \times 10^{12}$ \\
%    Energy range of $e^{+}$ [MeV] & [200,275] \\
%    Acceptance of the $e^{+}$ & 0.121 \\
%    Track reconstruction efficiency & 0.9 \\
    Total number of positrons &  $0.57\times 10^{12}$ \\
    Effective analyzing power &  0.42 \\
    \hline
    Statistical uncertainty on $\omega_{a}$ [ppb]  &  450 \\
    Statistical uncertainty on $\omega_{p}$ [ppb]  &  100 \\
    \hline
    Uuncertainties on $a_{\mu}$ [ppb]  & 460 (stat.) \\ 
                                                            & $<70$ (syst.) \\
    \hline
    Uncertainties on EDM [$10^{-21}~e\cdot$cm]  & 1.4 (stat.) \\
                                                                              & 0.36 (syst.) \\
    \hline
  \end{tabular}
%  \end{small}
  \end{center}
\end{table}

 A muon EDM will produce muon spin precession out of
the horizontal plane that is defined by the ideal muon orbit.
This can be seen from Eq.~\ref{eq:omega_tot}
where the second term is the EDM term that is perpendicular
to the $a_\mu$ term. Due to the fact that the EDM term generates
vertical motion of the spin, one can extract the EDM term
from the oscillation of the up and down asymmetry $\mathcal{A}_{UD}(t)$ in
number of positrons detected,
\begin{eqnarray}
\mathcal{A}_{UD}(t) = 
\frac{N^{up}(t) - N^{down}(t)}{N^{up}(t) + N^{down}(t)} =
\frac{PA_{EDM} \sin{(\omega_{tot}t+\phi)}}{1+ P A \cos{(\omega_{tot}t+\phi)}},
\end{eqnarray}
where $P$ and $A$ are the polarization of the muon and an effective analyzing power of muon decay, respectively. $A_{EDM}$ is an effective analyzing power associated with the EDM.
A simulated up-down asymmetry in the case of $d_{\mu}=1\times 10^{-20}~e\cdot\mbox{cm}$ 
is shown in Fig.~\ref{fig:Wiggle}~(right).
The estimated statistical sensitivity for EDM is $1.5\times 10^{-21}~e\cdot\mbox{cm}$ (See Table~\ref{tab:sensitivity}).

A major source of systematic uncertainties on EDM is detector misalignment with respect to
the plane of the muon storage. The alignment resolution is estimated as 
$0.36\times 10^{-21}~e\cdot\mbox{cm}$ from the resolution 
of the alignment monitor system made with an optical frequency comb technology.
Effects of axial electric field and radial magnetic field~\cite{Silenko:2017vvd}
are both less than $10^{-24}~e\cdot\mbox{cm}$, thus negligibly small.

%===============================================================
\subsection{Summary and prospects}\label{sec:Summary} 

%In summary, 
A new method of measuring $a_{\mu}$ and EDM of the muon is described.
The experiment utilizes a low-emittance muon beam prepared by 
reaccelerating thermal-energy muons created from laser-resonant 
ionization of muonium atoms. The low emittance muon beam allows use of a very
weak magnetic focusing and the selected low muon momentum (300~MeV/$c$) leads to use of a compact magnetic storage ring, 
instead of strong electric focusing at the magic momentum (3~GeV/$c$) used by previous and ongoing $g-2$ experiments.
A novel three dimensional spiral injection method with a pulsed magnetic kick
is adopted to store the muon beam in the storage ring efficiently.
The experiment reconstructs positron tracks from muons decay during their storage 
with a tracking detector consisted of silicon-strip sensors.

This experiment intends to reach statistical
uncertainties for $a_{\mu}$ of 460~ppb and for muon EDM of
$1.5\times 10^{-21}~e\cdot\mbox{cm}$, for an acquisition time
of $2 \times 10^7$ seconds. The statistical precision is comparable to
that of the BNL experiment. The EDM sensitivity is about two orders of magnitude
smaller than the BNL limits.
{Present estimates of systematic uncertainties on $a_{\mu}$ and EDM are
factor of seven and four smaller than the statistical uncertainties, respectively.
This experiment will test the 3~$\sigma$ deviation on $g-2$ reported by
the BNL experiment with significantly different and improved systematic uncertainties 
and search for new sources of T-violation in the muon with unprecedented sensitivity.

