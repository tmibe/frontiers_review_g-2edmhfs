\section{Theory of muon $g-2$, EDM, and muonium hyperfine splitting}

\def\hpz{\hphantom{0}} \def\hpzz{\hphantom{00}} \def\hph{\hphantom{-}}

\subsection{Standard Model Prediction for muon $g-2$}

The Standard Model (SM) prediction for the muon $g-2$
can be written as the sum of the QED, electroweak (EW), and
hadronic contributions:
%
\begin{align}
   a_\mu(\text{SM})
&=   a_\mu(\text{QED}) +  a_\mu(\text{EW})
   + a_\mu(\text{hadronic})~,
\end{align}
%
The QED and EW contributions are perturbatively calculable,
and are known to a sufficient precision.  On the contrary,
the hadronic contribution is not perturbatively calculable,
and hence it is the largest source of the uncertainty.
In Table \ref{table:a_mu_SM}, we show a breakdown of the
SM prediction, where the numbers are taken from Ref.~\cite{KNT18}.


\begin{table*}[thbp]
\caption{Breakdown of the SM prediction for the muon $g-2$,
together with the experimental value
and the deviation between the experimental value and 
the SM prediction.  The numbers are given in units of $10^{-10}$.
Note that the references in the last column are just sources
of the quoted numbers, and that there are too many
historically important references behind them to list in the table.}
%
\label{table:a_mu_SM}
%
\begin{center} \begin{tabular}[t]{l|ll}
\hline
  QED contributions 
 & \phantom{$-$}11~658~471.8971 (0.007)
 ~\hspace*{0.2cm}~ & Ref.~\cite{Aoyama-etal-2017} \\
%\hline
  EW contributions & \phantom{$-$00~000~0}15.36 (0.10)
 & Ref.~\cite{Gnendiger-etal} \\
%\hline
  hadronic contributions &  &\\
%\hline
~~~LO hadronic VP contributions  &
  \phantom{$-$00~000~}693.27~(2.46) & Ref.~\cite{KNT18}\\
%\hline
  ~~~NLO hadronic VP contributions &
  \phantom{00~000~00}$-9.82$ (0.04) &  Ref.~\cite{KNT18} \\
%\hline
  ~~~NNLO hadronic VP contributions & 
\phantom{$-$00~000~00}1.24 (0.01)  & Ref.~\cite{Kurz-etal-hadNNLO}\\
%\hline
  ~~~hadronic l-by-l contributions & 
\phantom{$-$00~000~00}9.8 (2.6) &
 Ref.~\cite{Nyffeler-LbL} \\
%\hline
  ~~~hadronic l-by-l NLO contributions & 
\phantom{$-$00~000~00}0.3 (0.2) &
 Ref.~\cite{Colangelo-etal-NLOLbL} \\
\hline \hline
  Standard Model prediction, $a_\mu(\text{SM})$& 
   \phantom{$-$}11~659~182.05~(3.56)   & Ref.~\cite{KNT18} \\
%\hline
  experimental value, $a_\mu(\text{exp})$
 & \phantom{$-$}11~659~209.1 (6.3) & Refs.~\cite{Bennett:2004pv,PDG17} \\
\hline \hline
 difference, $\Delta a_\mu$ 
 ($\equiv a_\mu(\text{exp}) - a_\mu(\text{SM})$) & 
\phantom{$-$0~000~00}27.05~(7.26)~, ~ 3.7 $\sigma$ & Ref.~\cite{KNT18} \\
\hline
\end{tabular} \end{center} \end{table*}




The QED contribution is known to 5-loop~\cite{AKN17}. 
The latest value is
%
\begin{align}
 a_\mu(\text{QED}) &=
  11 658 471.8971 (7)(17)(6)(72) \times 10^{-10} ~,
% &= (11 658 471.8971 \pm 0.007) \times 10^{-10}~,
\label{eq:a_mu_QED}
\end{align}
%
where the uncertainties are those from the lepton mass ratios,
the four-loop contributions, the five-loop contributions,
and the uncertainty in $\alpha$.  (In the value above, the
value of $\alpha$ obtained from the ${}^{87}\text{Rb}$ mass
determination experiment has been used).  Since the uncertainty
of the QED contribution is negligible compared to the uncertainty 
in the experimental value and the hadronic contributions,
the QED contribution does not seem to pose any problem.


The EW contribution is calculated up to and including
2-loop.  The most up-to-date value, which includes 
the Higgs boson mass, reads~\cite{Gnendiger}:
%
\begin{align}
 a_\mu(\text{EW}) &=  (15.36 \pm 0.10) \times 10^{-10} ~.
\label{eq:a_mu_EW}
\end{align}
%
Similarly to the QED contribution, the uncertainty of the EW
contribution is small enough.  Since this is a perturbatively
calculable quantity, this contribution is already understood
very well.

It is the hadronic contribution which is most intractable.
The hadronic contribution can be written as the sum of a 
few terms:
%
\begin{align}
   a_\mu(\text{had})
&= a_\mu^{\text{had, LO VP}} +  a_\mu^{\text{had, NLO VP}}
+  a_\mu^{\text{had, NNLO VP}} \nonumber \\
%
&\quad + a_\mu^{\text{had, LbL}} +  a_\mu^{\text{had, NLO LbL}}~,
%
\end{align}
%
where the terms in the first line of the right-hand side stand
for the LO, NLO and NNLO hadronic vacuum polarization (VP)
contributions, respectively, and the second line
the light-by-light (LbL) and the NLO LbL terms, respectively.
The VP contributions can be computed by using dispersion 
relations, while to evaluate the LbL terms we have to 
rely more or less on hadronic models.

The LO and NLO VP contributions evaluated in Ref.~\cite{KNT18}
are 
%
\begin{align}
 a_\mu^{\text{had, LO VP}}= (693.27 \pm 2.46) \times 10^{-10}~, 
\label{eq:a_mu_hadLOVP}
\end{align}
%
and
%
\begin{align}
a_\mu^{\text{had, NLO VP}}= (-9.82 \pm 0.04) \times 10^{-10}~, 
\label{eq:a_mu_hadNLOVP}
\end{align}
%
respectively, while the NNLO VP contribution computed 
in Ref.~\cite{Kurz-etal-hadNNLO} is
%
\begin{align}
a_\mu^{\text{had, NNLO VP}}= (1.24 \pm 0.01) \times 10^{-10}~.
\label{eq:a_mu_hadNNLOVP}
\end{align}
%
These VP contributions can be calculable via dispersion relations
by using experimental data of the reaction
$e^+e^- \to \text{hadrons}$ as input, 
which allows us to evaluate them at the precision of 
${\mathcal O}(1\%)$ or less.

As for the LbL contributions we use the value
as quoted in Ref.~\cite{KNT18}: 
%
%
\begin{align}
a_\mu^{\text{had, LbL}}= (9.8 \pm 2.6) \times 10^{-10}~.
\label{eq:a_mu_hadLbL}
\end{align}
%
This value has been obtained from the ``Glasgow consensus''
value~\cite{} with a correction from a recent reevaluation
of the axial vector particle exchange~\cite{}.
The value of $a_\mu^{\text{had, NLO LbL}}$ is computed
in Ref.~\cite{Colangelo-etal-NLOLbL}:
%
\begin{align}
a_\mu^{\text{had, NLO LbL}}= (0.3 \pm 0.2) \times 10^{-10}~.
\label{eq:a_mu_hadNLOLbL}
\end{align}

By summing up the values Eqs.~
(\ref{eq:a_mu_QED}), (\ref{eq:a_mu_EW}), (\ref{eq:a_mu_hadLOVP}),
(\ref{eq:a_mu_hadNLOVP}), (\ref{eq:a_mu_hadNNLOVP}),
(\ref{eq:a_mu_hadLbL}) and (\ref{eq:a_mu_hadNLOLbL}), 
we obtain, for the SM prediction, 
%
\begin{align}
 a_\mu(\text{SM}) &= (11 659 182.05 \pm 3.56) \times 10^{-10}~.
\label{eq:a_mu_SM}
\end{align}
%
This value should be compared with the experimentally measured
value, Eq.~(\ref{eq:a_mu_exp}).
The difference $\Delta a_\mu$ between Eqs.~(\ref{eq:a_mu_exp})
and (\ref{eq:a_mu_SM}) is
% 
\begin{align}
 \Delta a_\mu \equiv a_\mu(\text{exp}) - a_\mu(\text{SM})
= (27.05 \pm 7.26) \times 10^{-10}~,
\label{eq:delta_a_mu}
\end{align}
%
which means a 3.7 $\sigma$ discrepancy.

As seen from Table \ref{table:a_mu_SM}, 
the largest sources of the uncertainties are the
LO hadronic and LbL contributions.  It is therefore
extremely important to reduce the errors of these contributions
as much as possible.
In the following sections we discuss these contributions.

\subsection{LO hadronic contribution}

In this subsection we discuss the LO hadronic contribution.

The LO hadronic VP contribution can be evaluated by using
the dispersion integral,
%
\begin{align}
 a_\mu^{\text{had, LO VP}} =
  \frac{\alpha^2}{3\pi^2} \int_{m_\pi^2}^{\infty}
   \frac{\text{d}s}{s} R(s) K(s)~,
\label{eq:dispersion_relation_a_mu}
\end{align}
%
where $K(s)$ is a known kernel function and
$R(s)$ is the hadronic $R$-ratio,
%
\begin{align}
 R(s) =
  \frac{\sigma^0_{\text{had,} \gamma}(s)}{\sigma_{\text{pt}}(s)}
\equiv
  \frac{\sigma^0_{\text{had,} \gamma}(s)}{4\pi \alpha^2/(3s)}~.
\end{align}
%
The subscript `$\gamma$' means that the cross section
should include final state radiations (FSRs).
The superscript `0' is a reminder that the cross section
should not include VP radiative corrections.  We exclude
VP radiative corrections since we have to avoid
double-counting with higher order hadronic contributions.



To better understand the behavior of the kernel function $K(s)$,
it is useful to define a function $\hat{K}(s)$ by
%
\begin{align}
 \hat{K}(s) \equiv \frac{3s}{m_\mu^2}K(s)~.
\end{align}
%
The function $\hat{K}(s)$ is monotonically increasing
from the threshold $s=m_\pi^2$ to $s \to \infty$, 
with $\hat{K}(m_\pi^2)=0.40$, $\hat{K}(4m_\pi^2)=0.63$
and $\hat{K}(s) \to 1$ for $s \to \infty$.
The factor $1/s$ in the integrand of 
Eq.~(\ref{eq:dispersion_relation_a_mu}) and the
$s$-dependence of $K(s)$ as $K(s)= \hat{K}(s) m_\mu^2/(3s)
= {\mathcal O}(1) \times m_\mu^2/(3s)$,
enhances the low energy region, which makes the 
low energy hadronic data more important.


In Table \ref{table:contributions} we show some important
channels which give important contributions to the dispersive
integral, Eq.~(\ref{eq:dispersion_relation_a_mu}). 

\begin{table} 
\caption{Contributions from important channels to the 
dispersion integrals Eqs.~(\ref{eq:dispersion_relation_a_mu})
and (\ref{eq:dispersion_relation_DeltaAlpha}).
The numbers from the energy region $\sqrt{s} \leq 1.937$ GeV
include contributions from data as well as near-threshold
contributions to $2\pi$, $3\pi$ and $\pi^0\gamma$ channels.
The numbers in the table are taken from Ref.~\cite{KNT18}.
For a full table, see Ref.~\cite{KNT18}.}
%
\label{table:contributions}
%
\begin{center} \begin{tabular}{|l|c|c|}
\hline
channel        & $a_\mu^{\text{had, LO VP}} \times 10^{10}$
               & $\Delta \alpha_{\text{had}}^{(5)}(M_Z^2)\times 10^4$ \\
\hline
\multicolumn{3}{|c|}{Contributions from $\sqrt{s} \leq 1.937$ GeV} \\
\hline
$\pi^+\pi^-$   & $503.84 \pm 1.97$     & $\hpz34.27 \pm 0.12$ \\
$\pi^+\pi^-\pi^0$
               & $\hpz47.80 \pm 0.89$  & $\hpzz4.77 \pm 0.08$ \\
$K^+ K^-$      & $\hpz23.03 \pm 0.22$  & $\hpzz3.37 \pm 0.03$ \\
$\pi^+ \pi^- 2\pi^0$
               & $\hpz19.39 \pm 0.78$  & $\hpzz5.00 \pm 0.20$ \\
$2\pi^+ 2\pi^-$& $\hpz14.87 \pm 0.20$  & $\hpzz4.02 \pm 0.05$ \\
$K_S^0 K_L^0$  & $\hpz13.04 \pm 0.19$  & $\hpzz1.77 \pm 0.03$ \\
$\pi^0 \gamma$ & $\hpzz4.58 \pm 0.10$  & $\hpzz0.36 \pm 0.01$ \\
$KK\pi$        & $\hpzz2.71 \pm 0.12$  & $\hpzz0.89 \pm 0.04$ \\
$KK\pi\pi$     & $\hpzz1.93 \pm 0.08$  & $\hpzz0.75 \pm 0.03$ \\
$\hpzz\vdots$  &  $\vdots$  &  $\vdots$   \\
\hline 
\multicolumn{3}{|c|}{Contributions from
  $1.937 \leq \sqrt{s} \leq 11.199$ GeV} \\
\hline
Inclusive channel & $\hpz43.67 \pm 0.67$  & $\hpz82.82 \pm 1.05$ \\
\hline
\multicolumn{3}{|c|}{Narrow Resonance Contributions} \\
\hline
$J/\psi$          & $\hpzz6.26 \pm 0.19$  & $\hpzz7.07 \pm 0.22$ \\
$\psi'$           & $\hpzz1.58 \pm 0.04$  & $\hpzz2.51 \pm 0.06$ \\
$\Upsilon(1S-4S)$ & $\hpzz0.09 \pm 0.00$  & $\hpzz1.06 \pm 0.02$ \\
\hline 
\multicolumn{3}{|c|}{Contributions from $\sqrt{s} \geq 11.199$ GeV} \\
\hline
pQCD              & $\hpzz2.07 \pm 0.00$  & $124.79 \pm 0.10$ \\
\hline\hline
Total             & $693.27 \pm 2.46$     & $276.11 \pm 1.11$  \\
\hline
\end{tabular} \end{center} \end{table}

