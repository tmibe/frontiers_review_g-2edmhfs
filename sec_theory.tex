\section{Theory of muon $g-2$, EDM, and muonium hyperfine splitting}

\def\hpz{\hphantom{0}} \def\hpzz{\hphantom{00}} \def\hph{\hphantom{-}}

\subsection{Standard Model Prediction for muon $g-2$}

The Standard Model (SM) prediction for the muon $g-2$
can be written as the sum of the QED, electroweak (EW), and
hadronic contributions:
%
\begin{align}
   a_\mu(\text{SM})
&=   a_\mu(\text{QED}) +  a_\mu(\text{EW})
   + a_\mu(\text{hadronic})~,
\end{align}
%
The QED and EW contributions are perturbatively calculable,
and are known to a sufficient precision.  On the contrary,
the hadronic contribution is not perturbatively calculable,
and hence it is the largest source of the uncertainty.
In Table \ref{table:a_mu_SM}, we show a breakdown of the
SM prediction, where the numbers are taken from Ref.~\cite{KNT18}.


\begin{table*}[thbp]
\caption{Breakdown of the SM prediction for the muon $g-2$,
together with the experimental value
and the deviation between the experimental value and 
the SM prediction.  The numbers are given in units of $10^{-10}$.
Note that the references in the last column are just sources
of the quoted numbers, and that there are too many
historically important references behind them to list in the table.}
%
\label{table:a_mu_SM}
%
\begin{center} \begin{tabular}[t]{l|ll}
\hline
  QED contributions 
 & \phantom{$-$}11~658~471.8971 (0.007)
 ~\hspace*{0.2cm}~ & Ref.~\cite{Aoyama-etal-2017} \\
%\hline
  EW contributions & \phantom{$-$00~000~0}15.36 (0.10)
 & Ref.~\cite{Gnendiger-etal} \\
%\hline
  hadronic contributions &  &\\
%\hline
~~~LO hadronic VP contributions  &
  \phantom{$-$00~000~}693.27~(2.46) & Ref.~\cite{KNT18}\\
%\hline
  ~~~NLO hadronic VP contributions &
  \phantom{00~000~00}$-9.82$ (0.04) &  Ref.~\cite{KNT18} \\
%\hline
  ~~~NNLO hadronic VP contributions & 
\phantom{$-$00~000~00}1.24 (0.01)  & Ref.~\cite{Kurz-etal-hadNNLO}\\
%\hline
  ~~~hadronic l-by-l contributions & 
\phantom{$-$00~000~00}9.8 (2.6) &
 Ref.~\cite{Nyffeler-LbL} \\
%\hline
  ~~~hadronic l-by-l NLO contributions & 
\phantom{$-$00~000~00}0.3 (0.2) &
 Ref.~\cite{Colangelo-etal-NLOLbL} \\
\hline \hline
  Standard Model prediction, $a_\mu(\text{SM})$& 
   \phantom{$-$}11~659~182.05~(3.56)   & Ref.~\cite{KNT18} \\
%\hline
  experimental value, $a_\mu(\text{exp})$
 & \phantom{$-$}11~659~209.1 (6.3) & Refs.~\cite{Bennett:2004pv,PDG17} \\
\hline \hline
 difference, $\Delta a_\mu$ 
 ($\equiv a_\mu(\text{exp}) - a_\mu(\text{SM})$) & 
\phantom{$-$0~000~00}27.05~(7.26)~, ~ 3.7 $\sigma$ & Ref.~\cite{KNT18} \\
\hline
\end{tabular} \end{center} \end{table*}




The QED contribution is known to 5-loop~\cite{AKN17}. 
The latest value is
%
\begin{align}
 a_\mu(\text{QED}) &=
  11 658 471.8971 (7)(17)(6)(72) \times 10^{-10} ~,
% &= (11 658 471.8971 \pm 0.007) \times 10^{-10}~,
\label{eq:a_mu_QED}
\end{align}
%
where the uncertainties are those from the lepton mass ratios,
the four-loop contributions, the five-loop contributions,
and the uncertainty in $\alpha$.  (In the value above, the
value of $\alpha$ obtained from the ${}^{87}\text{Rb}$ mass
determination experiment has been used).  Since the uncertainty
of the QED contribution is negligible compared to the uncertainty 
in the experimental value and the hadronic contributions,
the QED contribution does not seem to pose any problem.


The EW contribution is calculated up to and including
2-loop.  The most up-to-date value, which includes 
the Higgs boson mass, reads~\cite{Gnendiger}:
%
\begin{align}
 a_\mu(\text{EW}) &=  (15.36 \pm 0.10) \times 10^{-10} ~.
\label{eq:a_mu_EW}
\end{align}
%
Similarly to the QED contribution, the uncertainty of the EW
contribution is small enough.  Since this is a perturbatively
calculable quantity, this contribution is already understood
very well.

It is the hadronic contribution which is most intractable.
The hadronic contribution can be written as the sum of a 
few terms:
%
\begin{align}
   a_\mu(\text{had})
&= a_\mu^{\text{had, LO VP}} +  a_\mu^{\text{had, NLO VP}}
+  a_\mu^{\text{had, NNLO VP}} \nonumber \\
%
&\quad + a_\mu^{\text{had, LbL}} +  a_\mu^{\text{had, NLO LbL}}~,
%
\end{align}
%
where the terms in the first line of the right-hand side stand
for the LO, NLO and NNLO hadronic vacuum polarization (VP)
contributions, respectively, and the second line
the light-by-light (LbL) and the NLO LbL terms, respectively.
The VP contributions can be computed by using dispersion 
relations, while to evaluate the LbL terms we have to 
rely more or less on hadronic models.

The LO and NLO VP contributions evaluated in Ref.~\cite{KNT18}
are 
%
\begin{align}
 a_\mu^{\text{had, LO VP}}= (693.27 \pm 2.46) \times 10^{-10}~, 
\label{eq:a_mu_hadLOVP}
\end{align}
%
and
%
\begin{align}
a_\mu^{\text{had, NLO VP}}= (-9.82 \pm 0.04) \times 10^{-10}~, 
\label{eq:a_mu_hadNLOVP}
\end{align}
%
respectively, while the NNLO VP contribution computed 
in Ref.~\cite{Kurz-etal-hadNNLO} is
%
\begin{align}
a_\mu^{\text{had, NNLO VP}}= (1.24 \pm 0.01) \times 10^{-10}~.
\label{eq:a_mu_hadNNLOVP}
\end{align}
%
These VP contributions can be calculable via dispersion relations
by using experimental data of the reaction
$e^+e^- \to \text{hadrons}$ as input, 
which allows us to evaluate them at the precision of 
${\mathcal O}(1\%)$ or less.

As for the LbL contributions we use the value
as quoted in Ref.~\cite{KNT18}: 
%
%
\begin{align}
a_\mu^{\text{had, LbL}}= (9.8 \pm 2.6) \times 10^{-10}~.
\label{eq:a_mu_hadLbL}
\end{align}
%
This value has been obtained from the ``Glasgow consensus''
value~\cite{} with a correction from a recent reevaluation
of the axial vector particle exchange~\cite{}.
The value of $a_\mu^{\text{had, NLO LbL}}$ is computed
in Ref.~\cite{Colangelo-etal-NLOLbL}:
%
\begin{align}
a_\mu^{\text{had, NLO LbL}}= (0.3 \pm 0.2) \times 10^{-10}~.
\label{eq:a_mu_hadNLOLbL}
\end{align}

By summing up the values Eqs.~
(\ref{eq:a_mu_QED}), (\ref{eq:a_mu_EW}), (\ref{eq:a_mu_hadLOVP}),
(\ref{eq:a_mu_hadNLOVP}), (\ref{eq:a_mu_hadNNLOVP}),
(\ref{eq:a_mu_hadLbL}) and (\ref{eq:a_mu_hadNLOLbL}), 
we obtain, for the SM prediction, 
%
\begin{align}
 a_\mu(\text{SM}) &= (11 659 182.05 \pm 3.56) \times 10^{-10}~.
\label{eq:a_mu_SM}
\end{align}
%
This value should be compared with the experimentally measured
value, Eq.~(\ref{eq:a_mu_exp}).
The difference $\Delta a_\mu$ between Eqs.~(\ref{eq:a_mu_exp})
and (\ref{eq:a_mu_SM}) is
% 
\begin{align}
 \Delta a_\mu \equiv a_\mu(\text{exp}) - a_\mu(\text{SM})
= (27.05 \pm 7.26) \times 10^{-10}~,
\label{eq:delta_a_mu}
\end{align}
%
which means a 3.7 $\sigma$ discrepancy.

As seen from Table \ref{table:a_mu_SM}, 
the largest sources of the uncertainties are the
LO hadronic and LbL contributions.  It is therefore
extremely important to reduce the errors of these contributions
as much as possible.
In the following sections we discuss these contributions.

\subsection{LO hadronic contribution}

In this subsection we discuss the LO hadronic contribution.

The LO hadronic VP contribution can be evaluated by using
the dispersion integral,
%
\begin{align}
 a_\mu^{\text{had, LO VP}} =
  \frac{\alpha^2}{3\pi^2} \int_{m_\pi^2}^{\infty}
   \frac{\text{d}s}{s} R(s) K(s)~,
\label{eq:dispersion_relation_a_mu}
\end{align}
%
where $K(s)$ is a known kernel function and
$R(s)$ is the hadronic $R$-ratio,
%
\begin{align}
 R(s) =
  \frac{\sigma^0_{\text{had,} \gamma}(s)}{\sigma_{\text{pt}}(s)}
\equiv
  \frac{\sigma^0_{\text{had,} \gamma}(s)}{4\pi \alpha^2/(3s)}~.
\end{align}
%
The subscript `$\gamma$' means that the cross section
should include final state radiations (FSRs).
The superscript `0' is a reminder that the cross section
should not include VP radiative corrections.  We exclude
VP radiative corrections since we have to avoid
double-counting with higher order hadronic contributions.



To better understand the behavior of the kernel function $K(s)$,
it is useful to define a function $\hat{K}(s)$ by
%
\begin{align}
 \hat{K}(s) \equiv \frac{3s}{m_\mu^2}K(s)~.
\end{align}
%
The function $\hat{K}(s)$ is monotonically increasing
from the threshold $s=m_\pi^2$ to $s \to \infty$, 
with $\hat{K}(m_\pi^2)=0.40$, $\hat{K}(4m_\pi^2)=0.63$
and $\hat{K}(s) \to 1$ for $s \to \infty$.
The factor $1/s$ in the integrand of 
Eq.~(\ref{eq:dispersion_relation_a_mu}) and the
$s$-dependence of $K(s)$ as $K(s)= \hat{K}(s) m_\mu^2/(3s)
= {\mathcal O}(1) \times m_\mu^2/(3s)$,
enhances the low energy region, which makes the 
low energy hadronic data more important.


In Table \ref{table:contributions} we show some channels
which give important contributions to the dispersive
integral, Eq.~(\ref{eq:dispersion_relation_a_mu}). 
As clearly seen from the table, the contribution 
from the $\pi^+\pi^-$ channel completely dominates over
the others in the mean value as well as in the uncertainty.
It is therefore extremely important to use good experimental
data as input to the dispersive integral.


\begin{table} 
\caption{Contributions from important channels to the 
dispersion integrals Eqs.~(\ref{eq:dispersion_relation_a_mu})
and (\ref{eq:dispersion_relation_DeltaAlpha}).
The numbers from the energy region $\sqrt{s} \leq 1.937$ GeV
include contributions from data as well as near-threshold
contributions to $2\pi$, $3\pi$ and $\pi^0\gamma$ channels.
The numbers in the table are taken from Ref.~\cite{KNT18}.
For a full table, see Ref.~\cite{KNT18}.}
%
\label{table:contributions}
%
\begin{center} \begin{tabular}{|l|c|c|}
\hline
channel        & $a_\mu^{\text{had, LO VP}} \times 10^{10}$
               & $\Delta \alpha_{\text{had}}^{(5)}(M_Z^2)\times 10^4$ \\
\hline
\multicolumn{3}{|c|}{Contributions from $\sqrt{s} \leq 1.937$ GeV} \\
\hline
$\pi^+\pi^-$   & $503.84 \pm 1.97$     & $\hpz34.27 \pm 0.12$ \\
$\pi^+\pi^-\pi^0$
               & $\hpz47.80 \pm 0.89$  & $\hpzz4.77 \pm 0.08$ \\
$K^+ K^-$      & $\hpz23.03 \pm 0.22$  & $\hpzz3.37 \pm 0.03$ \\
$\pi^+ \pi^- 2\pi^0$
               & $\hpz19.39 \pm 0.78$  & $\hpzz5.00 \pm 0.20$ \\
$2\pi^+ 2\pi^-$& $\hpz14.87 \pm 0.20$  & $\hpzz4.02 \pm 0.05$ \\
$K_S^0 K_L^0$  & $\hpz13.04 \pm 0.19$  & $\hpzz1.77 \pm 0.03$ \\
$\pi^0 \gamma$ & $\hpzz4.58 \pm 0.10$  & $\hpzz0.36 \pm 0.01$ \\
$KK\pi$        & $\hpzz2.71 \pm 0.12$  & $\hpzz0.89 \pm 0.04$ \\
$KK\pi\pi$     & $\hpzz1.93 \pm 0.08$  & $\hpzz0.75 \pm 0.03$ \\
$\hpzz\vdots$  &  $\vdots$  &  $\vdots$   \\
\hline 
\multicolumn{3}{|c|}{Contributions from
  $1.937 \leq \sqrt{s} \leq 11.199$ GeV} \\
\hline
Inclusive channel & $\hpz43.67 \pm 0.67$  & $\hpz82.82 \pm 1.05$ \\
\hline
\multicolumn{3}{|c|}{Narrow Resonance Contributions} \\
\hline
$J/\psi$          & $\hpzz6.26 \pm 0.19$  & $\hpzz7.07 \pm 0.22$ \\
$\psi'$           & $\hpzz1.58 \pm 0.04$  & $\hpzz2.51 \pm 0.06$ \\
$\Upsilon(1S-4S)$ & $\hpzz0.09 \pm 0.00$  & $\hpzz1.06 \pm 0.02$ \\
\hline 
\multicolumn{3}{|c|}{Contributions from $\sqrt{s} \geq 11.199$ GeV} \\
\hline
pQCD              & $\hpzz2.07 \pm 0.00$  & $124.79 \pm 0.10$ \\
\hline\hline
Total             & $693.27 \pm 2.46$     & $276.11 \pm 1.11$  \\
\hline
\end{tabular} \end{center} \end{table}


In Table \ref{table:comparison}, we show the difference
between the results of the two groups.  Although they 
agree well in the total numbers, the individual channels
do show differences.  These differences come from the
fact that the two groups use different method to combine
experimental data.


\begin{table*} 
\caption{Differences between the KNT18 analysis~\cite{KNT18} 
and the DHMZ17 analysis~\cite{DHMNZ17}, extracted from 
Table 5 of Ref.~\cite{KNT18}. 
Similarly to Table~\ref{table:contributions}, the numbers
from the region $\sqrt{s} \leq 2$ GeV
include contributions from data as well as near-threshold
contributions to $2\pi$, $3\pi$ and $\pi^0\gamma$ channels
evaluated by using chiral perturbation theory.
Note that although the default transition point
between the sum of exclusive channels and the inclusive
measurement is 1.937 GeV in Ref.~\cite{KNT18}, in this 
table we take the transition point at 1.8 GeV for comparison.}
%
\label{table:comparison}
%
\begin{center} \begin{tabular}{|l|c|c|c|}
\hline
channel        & KNT18~\cite{KNT18}    & DHMZ17~\cite{HLMNT} & Diff \\
\hline
\multicolumn{4}{|c|}{Contributions from $\sqrt{s} \leq 2$ GeV} \\
\hline
$\pi^+\pi^-$   &    $503.74 \pm 1.96$  &    $507.14 \pm 2.58$    & $-3.40$ \\
$\pi^+\pi^-\pi^0$
               & $\hpz47.70 \pm 0.89$  & $\hpz46.20 \pm 1.45$ & $\hph1.50$ \\
$K^+ K^-$      & $\hpz23.00 \pm 0.22$  & $\hpz22.81 \pm 0.41$ & $\hph0.19$ \\
$\pi^+ \pi^- 2\pi^0$
               & $\hpz18.15 \pm 0.74$  & $\hpz18.03 \pm 0.54$ & $\hph0.12$ \\
$2\pi^+ 2\pi^-$& $\hpz13.99 \pm 0.19$  & $\hpz13.68 \pm 0.31$ & $\hph0.31$\\
$K_S^0 K_L^0$  & $\hpz13.04 \pm 0.19$  & $\hpz12.82 \pm 0.24$ & $\hph0.22$ \\
$\pi^0 \gamma$ & $\hpzz4.58 \pm 0.10$  & $\hpzz4.29 \pm 0.10$ & $\hph0.29$ \\
$KK\pi$        & $\hpzz2.44 \pm 0.11$  & $\hpzz2.45 \pm 0.15$ & $-0.01$ \\
$KK\pi\pi$     & $\hpzz0.86 \pm 0.05$  & $\hpzz0.85 \pm 0.05$ & $\hph0.01$ \\
$\hpzz\vdots$  &  $\vdots$  &  $\vdots$   & \\
\hline 
\multicolumn{4}{|c|}{Narrow Resonance Contributions} \\
\hline
$J/\psi$         & $\hpzz6.26 \pm 0.19$  & $\hpzz6.28 \pm 0.07$ & $-0.02$ \\
$\psi'$          & $\hpzz1.58 \pm 0.04$  & $\hpzz1.57 \pm 0.03$ & $\hph0.01$ \\
$\Upsilon(1S-4S)$& $\hpzz0.09 \pm 0.00$  & $-$ & $\hph0.09$ \\
\hline 
\multicolumn{4}{|c|}{Contributions by energy region} \\
\hline
  $1.8 \leq \sqrt{s} \leq 3.7$ GeV
       & $\hpz34.54 \pm 0.56$  & $\hpz33.45 \pm 0.65$ & $\hph1.09$\\
  $3.7 \leq \sqrt{s} \leq 5.0$ GeV
       & $\hpzz7.33 \pm 0.11$  & $\hpzz7.29 \pm 0.03$ & $\hph0.04$\\
  $5.0 \leq \sqrt{s} \leq 9.3$ GeV
       & $\hpzz6.62 \pm 0.10$  & $\hpzz6.86 \pm 0.04$ & $-0.24$\\
  $9.3 \leq \sqrt{s} \leq 12.0$ GeV
       & $\hpzz1.12 \pm 0.01$  & $\hpzz1.21 \pm 0.01$ & $-0.09$\\
  $12.0 \leq \sqrt{s} \leq 40.0$ GeV
       & $\hpzz1.64 \pm 0.00$  & $\hpzz1.64 \pm 0.00$ & $\hph0.00$\\
  $> 40.0$ GeV
       & $\hpzz0.16 \pm 0.00$  & $\hpzz0.16 \pm 0.00$ & $\hph0.00$\\
\hline\hline
Total           & $693.3 \pm 2.5$     & $693.1 \pm 3.4$ & $\hph0.2$ \\
\hline
\end{tabular} \end{center} \end{table*}

\subsection{Muon $g-2$ in Physics beyond the SM}

In general, the effect of BSM physics on the muon $g-2$
can be described in terms of a low-energy effective Lagrangian
if the scale of the BSM physics is high enough compared to the
muon mass.   After integrating out heavy fields in the
BSM physics, we obtain the term below in the effective Lagrangian:
%
\begin{align}
 {\mathcal L} = \frac{e}{4 m_\mu}
 \frac{m_\mu^2}{\Lambda_{\text{NP}}^2} 
 \overline{\psi_L} ~ \sigma_{\mu\nu} F^{\mu\nu} \psi_R + {\text{h.\ c.\ }},
\label{eq:effL_g-2}
\end{align}
%
where $\psi$ is the field for the muon and
$\Lambda_{\text{NP}}$ is the scale of the new physics\footnote{
The $\Lambda_{\text{NP}}$-dependence of Eq.~(\ref{eq:effL_g-2})
is determined from the following discussion.  The
operators appearing on the right-hand side of Eq.~(\ref{eq:effL_g-2})
are chirality-flipping operators which connect the right-handed
fermion to the left-handed one, and hence the couplings in
front of the operators must be something which flips the chirality.
Since the muon mass flips the chirality, in 
Eq.~(\ref{eq:effL_g-2}) we naively assume that the operators
are multiplied by one factor of $m_\mu$.  To match the dimension,
we further need to multiply the operators by a quantity which
has a mass dimension of $[\text{mass}]^{-2}$.  
We take this as $\Lambda^{-2}_{\text{NP}}$.

Of course this is just a naive discussion, and the chirality
may be flipped by another chirality-flipping parameter generated
by new physics.  In such a case, the new physics contribution 
to $a_\mu$ may be proportional to $1/\Lambda_{\text{NP}}$,
not $1/\Lambda^2_{\text{NP}}$.  See e.g.\ Ref.~\cite{Czarnecki:2001pv}
for the models in which the latter possibility is realized.}.
In the non-relativistic limit, the effective interaction
above is equivalent to the contribution $\delta V_{\text{NP}}$
to the potential below
%
\begin{align}
 \delta V_{\text{NP}} = - \vec{\mu}_{\text{NP}} \cdot \vec{B}~,
\end{align}
%
where $\vec{\mu}_{\text{NP}}$ is the new physics
contribution to the magnetic moment of the muon and
$\vec{B}$ the magnetic field,
and hence the operator in Eq.~(\ref{eq:effL_g-2}) is
responsible for the muon $g-2$.
The contribution $a_\mu(\text{NP})$ to the muon $g-2$ from
the new physics is related to $\Lambda_{\text{NP}}$ as
%
\begin{align}
 a_\mu(\text{NP}) =  \frac{m_\mu^2}{\Lambda_{\text{NP}}^2} ~.
\end{align}
%
The scale $\Lambda_{\text{NP}}$ can be determined 
by assuming that the above contribution is responsible
for the non-zero value of $\delta a_\mu$.  For example,
if we take the evaluation of HLMNT11
in Eq.~(\ref{eq:chap2_tmp11}), we obtain
%
\begin{align}
 \Lambda_{\text{NP}} = 1.8 - 2.6 ~ (\text{TeV})~.
\end{align}
%
Note that this is a very rough estimate.
There are lots of BSM models which give contribution
to $a_\mu$.  
The most typical and well-studied model is the
supersymmetric (SUSY) models.  
To get a rough idea of the size of the SUSY contribution,
let us take the minimal SUSY SM as an example. 
In the MSSM, the leading SUSY contribution comes from
one-loop chargino-sneutrino and the neutralino-smuon diagrams.  
If we assume that all the SUSY particles have the
common mass $\tilde{m}$, the one-loop SUSY contribution
$a_\mu(\text{SUSY})$ is roughly given as~\cite{Czarnecki:2001pv}
%
\begin{align}
 a_\mu(\text{SUSY}) = (\text{sgn} \mu)  \times 13 \times 10^{-10}
\times \left( \frac{100 ~\text{GeV}}{\tilde{m}}\right)^2 \tan\beta~,
\end{align}
%
where $\tan\beta = \langle H_u \rangle/ \langle H_d \rangle$ 
is the ratio of the vacuum expectation values of the two Higgs
fields in the MSSM.  Note that e.g.\ for 
$(\tilde{m}, \tan\beta, \text{sgn}\mu) 
= ({\mathcal O}(300) ~\text{GeV}, {\mathcal O}(10), +1)$,
the size of $a_\mu(\text{SUSY})$ is precisely the right
range to explain the deviation, Eq.~(\ref{eq:chap2_tmp11}).  
The muon $g-2$ therefore gives a very important probe
of new physics, and it is very important to more firmly 
establish the currently observed deviation in the muon $g-2$.





\section{Physics of Muon EDM}

In this section we discuss physics of the muon electric dipole
moment (EDM) in the Standard Model (SM) and in models beyond
the SM.  For more detailed discussions on this topic, 
see e.g.\ Refs.~\cite{Barr:1988mc, Bernreuther:1990jx, Pospelov:2005pr, 
Fukuyama:2012np, Fukuyama:2015yya} and references therein. 
Most of the discussions on the theoretical predictions for
the electron EDM from various models given in
Refs.~\cite{Barr:1988mc, Bernreuther:1990jx, Pospelov:2005pr,
Fukuyama:2012np, Fukuyama:2015yya} also apply to the
muon EDM with obvious modifications like $m_e \to m_\mu$. 

\subsection{Standard Model Prediction for Muon EDM}

In this subsection we discuss the
prediction for the muon electric dipole moment (EDM)
from the Standard Model (SM).

In the minimal SM, the only source of CP violation is
the Kobayashi-Maskawa (KM) phase~\cite{Kobayashi:1973fv}.
Since this phase factor appears only in the coupling between
the quark and the $W$ boson, in order for the muon to receive a 
non-vanishing CP violation, we need at least two-loop
diagrams.  Actually, the two-loop diagrams do not add up to
a net CP violation since in these diagrams the 
Cabbibo-Kobayashi-Maskawa (CKM) matrix elements $V_{ij}$
$(i,j= 1,\cdots, 3)$ always appear in the combination $|V_{ij}|^2$.  
It follows that we need at least three-loop diagrams to
have a non-vanishing muon EDM.  
Surprisingly, it is shown in Ref.~\cite{Pospelov:1991zt} that
the sum of the three-loop diagrams does not yield a non-zero
electron EDM.  The authors of Ref.~\cite{Pospelov:1991zt}
also suggest that there may be a non-vanishing contribution
to the electron EDM at four-loop level.
These conclusions apply to the muon EDM as well.
In Ref.~\cite{Booth:1993af}, the size of the four-loop
contribution to the electron EDM $d_e(\text{SM})$ is estimated as 
%
\begin{align}
 d_e(\text{SM}) \sim 8 \times 10^{-41} e {\text{cm}}~.
\end{align}
%
By using this estimate, the SM prediction for
the muon EDM $d_\mu(\text{SM})$ is estimated to be
%
\begin{align}
 d_\mu(\text{SM}) \sim \frac{m_\mu}{m_e} d_e(\text{SM})
\sim 2 \times 10^{-38} e {\text{cm}}~.
\end{align}
%
This is much smaller than the uncertainty in the current experimental 
value of the muon EDM $d_\mu(\text{exp})$~\cite{Bennett:2008dy},
%
\begin{align}
 d_\mu(\text{exp}) = (-0.1\pm 0.9) \times 10^{-19}  e {\text{cm}}~,
\end{align}
%
and the aimed sensitivity of this experiment,
$\delta d_\mu(\text{J-PARC exp.}) \sim 10^{-21} ~ e {\text{cm}}$,
and hence the discovery of non-vanishing $d_\mu$ immediately
means that it is a hint of new physics beyond the SM.


\subsection{Muon EDM in Physics beyond the SM}

Similarly to the muon $g-2$ case, the effect of BSM physics
on the muon EDM can also be described in terms of the effective
Lagrangian when the scale of the BSM physics heavy enough
compared to $m_\mu$.
After integrating out the heavy BSM fields,
we obtain the term below in the effective Lagrangian:
%
\begin{align}
 {\mathcal L} = - \frac{i}{2} d_\mu^{\text{NP}}
 \overline{\psi} \sigma_{\mu\nu} \gamma_5 F^{\mu\nu} \psi ~,
\label{eq:effL_EDM}
\end{align}
%
where $\psi$ is the field for the muon, and the coefficient
$d_\mu^{\text{NP}}$ is the BSM contribution to the muon EDM.
In the nonrelativistic limit, this operator reduces to the
correction $\delta V_{\text{NP}}$
to the potential from the new physics:
%
\begin{align}
 \delta V_{\text{NP}} 
  = - d_\mu^{\text{NP}} \vec{\sigma} \cdot \vec{E}~,
\end{align}
%
where $\vec{\sigma}$ is the vector of Pauli matrices
for the spin $\vec{s}$ of the muon (namely, 
$\vec{s}= \frac12 \vec{\sigma}$) and $\vec{E}$ is
the external electric field. 

We should note the similarity of Eqs.~(\ref{eq:effL_g-2}) 
and (\ref{eq:effL_EDM}).  Motivated by this similarity, it 
is interesting to naively estimate the expected magnitude
of $d_\mu^{\text{NP}}$ by using the above expressions.
Assuming that the coefficients in Eqs.~(\ref{eq:effL_g-2}) 
and (\ref{eq:effL_EDM}) are about the same magnitude for
some reason, we obtain
%
\begin{align}
  \left| - \frac{i}{2} d_\mu^{\text{NP}} \right| \simeq 
   \left| \frac{e}{4m_\mu} a_\mu(\text{NP}) \right|~.
\end{align}
%
Substituting the value of $a_\mu(\text{NP})$ into
the above equation assuming that the deviation 
Eq.~(\ref{eq:chap2_tmp11}) entirely comes from new physics,
we obtain
%
\begin{align}
 |d_\mu^{\text{NP}}| \simeq 2.5 \times 10^{-22} e {\text{cm}}~
\text{ for } 
 a_\mu({\text{NP}}) \simeq 25 \times 10^{-10} 
~~~~~~(\text{naive estimate}).
\end{align}
%
Note that the estimated value of $|d_\mu^{\text{NP}}|$ is
in the ballpark of the expected sensitivity of this experiment,
$\delta d_\mu(\text{J-PARC exp.}) \sim 10^{-21} ~ e {\text{cm}}$.
If there is another enhancement factor of $\gtrsim 4$
in $d_\mu^{\text{NP}}$ which is not included in the above estimate,
the new physics contribution to the muon EDM may be observable.

