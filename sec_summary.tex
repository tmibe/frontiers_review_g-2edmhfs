\section{Summary}

In this article, we have given a brief overview of the
current status of theory and experiments on the muon $g-2$,
muon EDM and muonium hyperfine splitting, and a prospect
expected in the near future.  In particular, we have put
some emphasis on the activities in J-PARC.

Currently, there is a more-than 3 $\sigma$ discrepancy
between the experimental value of the muon $g-2$ and the
SM prediction for it.  The current experimental precision
of the muon $g-2$ is 0.5 ppm, while the uncertainty of
the SM prediction for it is 0.3 ppm.  There are two
experiments which will improve the experimental value:
one is at J-PARC, and the other at Fermilab.  Both
experiments aim to reduce the uncertainty by a factor
of 4, compared to the current value.  Although the Fermilab
experiment uses the same technique as that used in the BNL
experiment, the J-PARC experiment relies on a completely
different method, and hence is an independent check of the
BNL and Fermilab experiments.  

The SM prediction for the muon $g-2$ will also be improved
by new data of $e^+ e^- \to \pi^+\pi^-$ from Belle II, CMD-3
and SND.  Improvements of the evaluation of the light-by-light
contribution and the lattice calculations can also be expected.

Observation of a non-vanishing muon EDM immediately means
a signal of new physics beyond the SM.  In the J-PARC
experiment, it is expected to improve the experimental
upper bound of the muon EDM by a factor of 60.

The muonium HFS is an indispensable input parameter to
derive an experimental value of $a_\mu$ from the 
anomalous spin precession frequency $\omega_a$.
Currently, the indirectly observed value of the 
magnetic moment ratio $\mu_\mu/\mu_p$ is used in the
BNL paper, but ultimately this should be replaced by
a directly measured value of $\mu_\mu/\mu_p$.  In the
on-going experiment MuSEUM, it is possible to obtain
the directly measured value of $\mu_\mu/\mu_p$ to the
accuracy of x.xx ppm, which makes it possible to
replace the current indirectly measured value of
$\mu_\mu/\mu_p$ in the near future.  This will put the
status of the reported muon $g-2$ anomaly on a firmer
ground. 

All of the above quantities serve as useful probes of new
physics beyond the SM.  In particular, since there is a
more than 3 $\sigma$ anomaly reported on the muon $g-2$,
it is extremely important to study the muon sector very 
carefully.  Since no new physics beyond the
SM has been observed at experiments at high energies 
like the LHC, the importance of the precision physics 
in the muon sector has become higher than ever.  We hope that
these efforts will shed light on the existence of physics
beyond the SM.
















 






