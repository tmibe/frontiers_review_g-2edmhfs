\section{Introduction}

The anomalous magnetic moment of the muon, also called
the muon $g-2$, is one of the most precisely measured
quantities in particle physics.  The current experimental
value quoted in Ref.~\cite{PDG} reads
%
\begin{align}
 a_{\mu}^{\text{exp}}
= 11 \, 659 \, 209.1 (5.4) (3.3) \times 10^{-10} ~,
\label{eq:a_mu_exp}
\end{align}
%
where the first and second errors are the statistical and
systematic errors, respectively.  The number above is a
world average of past experiments, but it is almost completely
dominated by the result from the BNL E821
experiment \cite{Bennett:2006fi}.  
As we will discuss below, the total experimental uncertainty
will be reduced by a factor of four by two experiments: one is 
the J-PARC E34 experiment \cite{TDRsummarypaper}, 
and the other the Fermilab E989 experiment \cite{Grange:2015fou}.

The muon $g-2$ is not only precisely measured but also accurately
calculable, which makes this quantity very important.
The Standard Model (SM) prediction for the muon $g-2$ quoted in
Ref.~\cite{KNT18} reads
%
\begin{align}
 a_{\mu}^{\text{SM}}
= (11 \, 659 \, 182.05 \pm 3.56) \times 10^{-10} ~,
\label{eq:a_mu_SM_KNT18}
\end{align}
%
By comparing Eqs.~(\ref{eq:a_mu_exp}) and (\ref{eq:a_mu_SM_KNT18}),
we can test the SM.  Currently there is a discrepancy of 
3.7 $\sigma$ between Eqs.~(\ref{eq:a_mu_exp}) and (\ref{eq:a_mu_SM_KNT18}), 
which could be due to the contribution from physics beyond the SM. 

The electric dipole moment (EDM) of the muon is another interesting
observable.  In the SM, a non-vanishing lepton EDM of charged leptons
comes only from four loop (and higher order) diagrams.  It follows
that the SM prediction for the muon EDM $d_\mu$ is extremely small,
of the order of ${\mathcal O}(10^{-38}) e \text{cm}$.
This number should be compared with the current experimental
value,
%
\begin{align}
 d_\mu(\text{exp}) = (0.0 \pm 0.9) \times 10^{-19} \ e \text{ cm} ~,
\end{align}
%
and the aimed sensitivity of the J-PARC $g-2$/EDM
experiment \cite{TDRsummarypaper},
%
\begin{align}
 \delta d_\mu(\text{J-PARC exp}) = 1.5 \times 10^{-21} \ e \text{ cm} ~.
\end{align}
%
This means that an observation of a non-zero muon EDM immediately
implies the existence of new physics beyond the SM.  

Another closely related observable is the ground-state hyperfine
splitting (HFS) of the muonium.  The value of the muonium hyperfine
splitting can be used to extract the muon-proton magnetic
moment ratio, $\mu_\mu/\mu_p$.  As discussed later in this review,
the value of this ratio is used as an input parameter to obtain
the value Eq.~(\ref{eq:a_mu_exp}) in the BNL E821 experiment.  
More explicitly, the BNL paper obtains the value of $a_\mu$
from the equation,
%
\begin{align}
 a_\mu(\text{exp}) =
  \frac{\omega_a/\omega_p}{\mu_\mu/\mu_p - \omega_a/\omega_p} ~, 
\end{align}
%
where $\omega_a$ is the muon anomalous spin precession frequency,
while $\omega_p$ is the free proton NMR frequency.
Currently, the value obtained from an indirect measurement of 
$\mu_\mu/\mu_p$ is used, assuming that the SM is the correct
theory at low energies.  However, it is more desirable to use
a directly measured value of $\mu_\mu/\mu_p$ obtained without
assuming the SM since we are trying to search for physics
beyond the SM.  One of the main
purposes of the MuSEUM experiment is to provide such a direct
measurement with the highest precision ever, as discussed later.  

The content of this paper is as follows.  In Section 2, we
give the status of the theoretical predictions for the muon $g-2$,
EDM and the muonium HFS.  In Section 3, we discuss the J-PARC
E34 experiment, which measures the $g-2$ and EDM of the muon.
In Section 4, we introduce the MuSEUM experiment.  In Section 5,
we give summary and conclusions.





